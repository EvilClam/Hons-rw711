\documentclass[twocolumn,a4paper,10pt]{article}
\usepackage{amsmath}
\usepackage{float}
\usepackage{graphicx}
\usepackage[toc, acronym]{glossaries}
\usepackage{enumitem}
\usepackage{hyperref}
\hypersetup{colorlinks=true, linkcolor=black}
\setdescription{leftmargin=\parindent,labelindent=\parindent}
\newacronym{lgca}{LGCA}{Lattice Gas Cellular Automata}
\makeglossaries
\auhtor{Mr. S. Schreiber}
\date{\today}
\title{Flow modelling with lattice gas cellular automata}
\begin{document}
\maketitle
\begin{abstract}
The use of \acrshort{lgca} in flow modelling has become increasingly popular over the last couple of decades. In this article we'll be discussing three models of how one can achieve this i.e. HPP, FHP and FCHC. The following will be discussed for each model: assumptions, the collision rules, the propagation rules, calculation of the mass and momentum density, time complexity analysis, applications, defects and does it obey the desired hydrodynamics equation (Navier Stokes) in the macroscopic limit. 
\end{abstract}
\section{Introduction}

\section{History}
\section{\acrfull{lgca}}
\subsection{HPP}
\subsubsection{Assumptions}
\subsubsection{Collision rules}
\subsubsection{Propagation rules}
\subsubsection{Mass and Momentum Densities}
\subsubsection{Time complexity analysis}
\subsubsection{Applications}
\subsubsection{defects}
\subsection{FHP}
\subsubsection{Assumptions}
\subsubsection{Collision rules}
\subsubsection{Propagation rules}
\subsubsection{Mass and Momentum Densities}
\subsubsection{Time complexity analysis}
\subsubsection{Applications}
\subsubsection{defects}
\subsection{FCHC}
\subsubsection{Assumptions}
\subsubsection{Collision rules}
\subsubsection{Propagation rules}
\subsubsection{Mass and Momentum Densities}
\subsubsection{Time complexity analysis}
\subsubsection{Applications}
\subsubsection{defects}
\end{document}