\documentclass[a4paper, 12pt]{article}
\date{27 February 2014}
\author{Shaun Schreiber \\ 16715128}
\title{Tutorial #3}
\usepackage{float}
\usepackage{tikz}
\usepackage{graphicx}
\usetikzlibrary{arrows, automata}
\begin{document}
\maketitle
\section*{Question 3a}
\begin{theorem}
L = \{ (ab)^{n}a^{k}| n > k, k \geq 0 \}\\
\mbox{Let }n = p + 1 \ k = p.\\
S = (ab)^{p + 1}a^{p}\\
\mbox{Case 1}\\\\
S = \underbrace{(ab)^{p}}_{x}\underbrace{(ab)}_{y} \underbrace{(a)^{p}}_{z}\\
S = xy^{i}z, i \geq 0.\\
\mbox{Let }i = 0, S = xz = (ab)^{p}a^{p}.\\
\mbox{Contradiction}
\\\\
\mbox{Case 2}\\\\
S = \underbrace{(ab)^{p}a}_{x}\underbrace{ba}_{y} \underbrace{(a)^{p-1}}_{z}\\
S = xy^{i}z, i \geq 0.\\
\mbox{Let }i = 0, S = xz = (ab)^{p}aa^{p - 1} = (ab)^{p}a^{p}.\\
\mbox{Contradiction}\\\\
\mbox{Case 3}\\\\
S = \underbrace{(ab)^{p - 1}}_{x}\underbrace{bab}_{y} \underbrace{a(a)^{p-1}}_{z}\\
S = xy^{i}z, i \geq 0.\\
\mbox{Let }i = 2, S = xyyz = (ab)^{p -1 }babbab(a)^{p}\\
\mbox{Contradiction}\\\\
\mbox{Case 4}\\\\
\mbox{Let $y$ only consist of a's}\\
S = \underbrace{(ab)^{p + 1}}_{x}\underbrace{(a)^{p}}_{y}}\\
S = xy^{i}, i \geq 0.\\
S = xyy\\
\mbox{Contradiction}\\\\
\end{theorem}



\section*{Question 3b}
\end{document}
